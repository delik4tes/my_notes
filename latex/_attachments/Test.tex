\documentclass[11pt,  a4paper]{article}
\pagestyle{empty} %Убирает номерацию страниц
\usepackage{amsmath,  amssymb,  amsfonts} %модули для математики
\usepackage{float}
\usepackage{enumerate}
\usepackage{hyperref}
\usepackage{fullpage}
\usepackage[margin = 1in]{geometry}
\usepackage[russian]{babel}
\usepackage{graphicx}
\usepackage[none]{hyphenat}
\usepackage{fancyhdr}
\parindent 0px
\setlength{\parindent}{4cm}
%\setlength{\parskip}{1cm}

% Офомрление страниц
\pagestyle{fancy}
\fancyhead{}
\fancyfoot{}
\fancyhead[L]{Title Here}
\fancyhead[R]{Egor Zhelagin}
\fancyhead[C]{Math} 


\title{My \LaTeX\ Document}
\author{Egor Zhelagin}
\date{\today} %Можно свою дату ввести произвольную


\def\eq1{y=\dfrac{x}{3x^2+x+1}}

\begin{document}
\maketitle

Hello world in \LaTeX\ document.\\
Let's ecamine the function $\eq1$

\begin{titlepage}
\begin{center}
\vspace*{1cm}
\Large{\textbf{IB Mathematics SL}}\\
\Large{\textbf{Internal Assessment}}\\
\vfill
\line(1,0){400}\\[1mm]
\huge{\textbf{This is a Sample Title}}\\[3mm]
\Large{\textbf{This is a Sample Subtitle}}\\[1mm]
\line(1,0){400}\\[1mm]
\vfill
\end{center}
\end{titlepage}

\tableofcontents
\thispagestyle{empty} %отделяет содержание на отдельнюю страницу
\clearpage

\setcounter{page}{1}

Common Math:
$3x+4$
$$3x+4$$
$$3x^3+4$$
$$3x^{4x+4}+4$$

subscripts
$$x_1$$
$$x_{12}$$
$$x_{1_2}$$
$$x_{1_{2_3}}$$
$$a_0,a_1,a_2,  \ldots,  a_{100}$$

Greek letters
$$\pi$$
$$\Pi$$
$$\alpha$$
$$A=\pi r^2$$ %Окружность

Trig functions 
$$ y=\sin x$$
$$y=\cos\theta$$
$$y=sin^{-1}x$$
$$y=\arcsin 1$$

Log Functions
$$z=\log x$$
$$z=\log_5 x$$
$$z=\log_{5^2} x^2$$
$$z=\ln x$$

Lim:
$$\lim\limits_{x\to a}f(x)$$

Int:
$$\int\sin x\,dx = -\cos x +C $$
$$\int_a^b f(x)$$
$$\displaystyle{\int \limits_{a}^{b} x^2 \,dx = \left[\frac{x^3}{3}\right]_{a}^{b}=\frac{b^3}{3}-\frac{a^3}{3}}$$\\

Roots
$$\sqrt[3]{2}$$
$$\sqrt{x^2+y^2}$$
$$\sqrt{1+\sqrt{x}}$$

Fractions:
$$\frac{2}{3}$$
About $\displaystyle \frac{2}{3}$ of the glass is full.\\[4pt] %Сколько пустых строчек отступить
About $\frac{2}{3}$ of the glass is full.
														%Если пустая строчка,  то следующая строчка красная строка
About $\frac{2}{3}$ of the glass is full. 

$$\dfrac{2}{3}$$
$$\frac{\sqrt{x+1}}{\sqrt{x+1}}$$\\
The functions $f(x)=(x-3)^2+\frac{1}{2}$ has domain $\mathrm{D}_f:(-\infty,\infty)$

The distributive property  states that $a(b+c)=ab+ac$ for all $a,b,c,\in\mathbb{R}$.\\[6pt]
The equivalence class of $a$ is $[a]$.\\
The set $a$ is defined to be $\{1,2,3\}$\\
The movie ticket costs $\$11.50$.\\
$$2\left(\frac{1}{x^2+1}\right)$$
$$2\left[\frac{1}{x^2+1}\right]$$
$$2\left\{\frac{1}{x^2+1}\right\}$$
$$2\left \langle\frac{1}{x^2+1}\right\rangle$$
$$2\left|\frac{1}{x^2+1}\right|$$
$$\left.\frac{dy}{dx}\right|_{x=1}$$

Tables:\\
\begin{tabular}{|c|c|c|c|c|c|} %Во второй скобке количество колонн (столбцов) & - разделят текст в колонках
\hline
$x$ & 1 & 2 & 3 & 4 & 5\\ \hline %горизонтальная линия
$f(x)$ & $\frac{1}{2}$ & 11 & 12 & 13 & 14\\ \hline
\end{tabular}

\vspace{1cm}

\begin{table}[H] %чтобы отображалось в нужном месте
\def\arraystretch{1.5}
\centering
\caption{These values represen  the function $f(x)$}
\begin{tabular}{|c||c|c|c|c|c|} %Во второй скобке количество колонн (столбцов) & - разделят текст в колонках
\hline
$x$ & 1 & 2 & 3 & 4 & 5\\ \hline %горизонтальная линия
$f(x)$ & $\frac{1}{2}$ & 11 & 12 & 13 & 14\\ \hline
\end{tabular}
\caption{These values represen  the function $f(x)$}
\end{table}

%c - центр
%l - слева
%r - справа
%p{} -параграф

\begin{table}[H] 
\def\arraystretch{1.5}
\centering
\caption{These values represen  the function $f(x)$}
\begin{tabular}{|l|p{2in}|} %
\hline
$f(x)$&$f'(x)$\\ \hline %горизонтальная линия
$x>0$& The function.\\ \hline
\end{tabular}
\end{table}

Arrays:

% & - ориентировка по цифре
\begin{align}
5x&^2\text{ place your words here}\\
5x&-12=3x-10
\end{align}

\begin{align*}
x=1
\end{align*}

\begin{align}
2x&=2\\
x&=1
\end{align}

\begin{enumerate}
\item pencil
\item calculator
\item notebook
	\begin{enumerate}
	\item notes
	\item homework
		\begin{enumerate}
		\item Math
		\item Phys
		\item IT
		\end{enumerate}
	\end{enumerate}
\end{enumerate}

\vspace{0.5cm}

\begin{itemize}
\item pencil
\item calculator
\item notebook
\end{itemize}

\vspace{0.5cm}

\begin{enumerate}[A.]
\item pencil
\item calculator
\item notebook
\end{enumerate}


\vspace{0.5cm}

\begin{enumerate}[i.]
\item pencil
\item calculator
\item notebook
\end{enumerate}


\vspace{0.5cm}

\begin{enumerate} \setcounter{enumi}{5}
\item pencil
\item calculator
\item notebook
\end{enumerate}

\vspace{0.5cm}

\begin{enumerate} \setcounter{enumi}{5}
\item[a)] pencil
\item[two] calculator
\item[4] notebook
\end{enumerate}

Work with text:\\
This will produce \textit{italic}.\\
This will produce \textbf{bold}.\\
This will produce \textsc{small caps}.\\
This will produce \texttt{typewriter font}.\\  %Лучше использовать для ссылок
This will produce \url{http://youtube.com}\\
This will produce \href{http://youtube.com}{Youtube}
\vspace{1cm}

Please check my blog\\

\begin{large}
Please check my blog\\
\end{large}

\begin{Large}
Please check my blog\\
\end{Large}

\begin{LARGE}
Please check my blog\\
\end{LARGE}

\begin{huge}
Please check my blog\\
\end{huge}

\begin{Huge}
Please check my blog\\
\end{Huge}

\begin{normalsize}
Please check my blog
\end{normalsize}

\begin{small}
Please check my blog
\end{small}

\begin{scriptsize}
Please check my blog
\end{scriptsize}

\begin{tiny}
Please check my blog
\end{tiny}

\vspace{1cm}

\begin{center}
Please check my blog
\end{center}

\begin{flushleft}
Please check my blog
\end{flushleft}

\begin{flushright}
Please check my blog
\end{flushright}

\section{Linear Functions}
	\subsection{Slope-Intercept Form}
		\subsubsection{Example 1}
		\subsubsection{Example 2}
	\subsection{Standard Form}
	\subsection{Point-Slope Form}
\section{Quadratic Functions}

\pagebreak

\textbf{Critical Thinking Questions}

\begin{figure}[h]
\centering
\includegraphics[scale=0.3]{iu}\\
\caption{This is a kinds of figures}
\label{fig:kinds}
\end{figure}

\begin{figure}[b] %H - где находится код,  t- сверху страницы,  b-снизу страницы
\includegraphics[scale=0.6]{iu}
\end{figure}


\begin{enumerate}
\item Is it possible for a sequence to converge to two different numbers? If so, give an example(see picture \ref{fig:kinds}). If not, explain why not.
\item Explain how to use partial sums to determine if a series converges or diverges. Give an example
\item Explain why $\int\limits_{1}^{\infty} f(x)\,dx$ and $\sum\limits_{n=1}^{\infty} a_n$ need not converge to the same value, even if they are both convergent.
\item  In your own words, explain the Alternating Series Remainder Theorem. How is this theorem useful?
\item Explain the difference between absolute and conditional convergence. Give an example of each.
\item The Ratio Test is inconclusive if $\displaystyle{\lim\limits_{n \to \infty} \left| \frac{a_{n+1}}{a_n} \right| =1}$. Give an example of one convergent series and one divergent series for which $\displaystyle{\lim\limits_{n \to \infty} \left| \frac{a_{n+1}}{a_n} \right| =1}$. Explain how you determined your examples.\footnote{This first footnote}
\end{enumerate}


\pagebreak
\begin{thebibliography}{2}

\bibitem{1}
Egor Zhelagin
	High School
\textit{Iternal Assessment}
Web: 27 may 2015

\bibitem{2}
\end{thebibliography}


\end{document}